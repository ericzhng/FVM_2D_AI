\documentclass[a4paper,12pt]{article}
\usepackage{amsmath}
\usepackage{amsfonts}
\usepackage{amssymb}
\usepackage{geometry}
\geometry{margin=1in}
\usepackage{parskip}
\usepackage{noto}

\begin{document}

\section*{2D Shallow Water Equations for Finite Volume Solver}

\subsection*{Governing Equations}
The 2D shallow water equations in conservative form are:

\[
\frac{\partial \mathbf{U}}{\partial t} + \frac{\partial \mathbf{F}}{\partial x} + \frac{\partial \mathbf{G}}{\partial y} = \mathbf{S}
\]

Where:
\begin{itemize}
    \item Conserved variables: \(\mathbf{U} = \begin{bmatrix} h \\ hu \\ hv \end{bmatrix}\)
    \begin{itemize}
        \item \(h\): water depth
        \item \(hu\): x-momentum
        \item \(hv\): y-momentum
    \end{itemize}
    \item Fluxes:
    \begin{itemize}
        \item \(\mathbf{F} = \begin{bmatrix} hu \\ hu^2 + \frac{1}{2}gh^2 \\ huv \end{bmatrix}\)
        \item \(\mathbf{G} = \begin{bmatrix} hv \\ huv \\ hv^2 + \frac{1}{2}gh^2 \end{bmatrix}\)
    \end{itemize}
    \item Source term: \(\mathbf{S} = \begin{bmatrix} 0 \\ -gh \frac{\partial z}{\partial x} \\ -gh \frac{\partial z}{\partial y} \end{bmatrix}\)
    \begin{itemize}
        \item \(g\): gravitational acceleration
        \item \(z\): bed elevation
    \end{itemize}
\end{itemize}

\subsection*{Riemann Problem}
For the Riemann problem, define a discontinuity across a cell interface:
\begin{itemize}
    \item Left state: \(\mathbf{U}_L = \begin{bmatrix} h_L \\ (hu)_L \\ (hv)_L \end{bmatrix}\)
    \item Right state: \(\mathbf{U}_R = \begin{bmatrix} h_R \\ (hu)_R \\ (hv)_R \end{bmatrix}\)
    \item Initial condition: Specify different \(h\), \(hu\), \(hv\) on either side of a line (e.g., \(x = 0\)).
\end{itemize}

\subsection*{Roe Flux}
The Roe flux for the interface between left and right states is:

\[
\mathbf{F}_{Roe} = \frac{1}{2} \left( \mathbf{F}(\mathbf{U}_L) + \mathbf{F}(\mathbf{U}_R) \right) - \frac{1}{2} \sum_{k=1}^3 |\tilde{\lambda}_k| \tilde{\mathbf{K}}_k \tilde{\alpha}_k
\]

Where:
\begin{itemize}
    \item \(\tilde{\lambda}_k\): eigenvalues of the Roe-averaged Jacobian
    \item \(\tilde{\mathbf{K}}_k\): eigenvectors
    \item \(\tilde{\alpha}_k\): wave strengths
    \item Roe averages:
    \begin{itemize}
        \item \(\tilde{u} = \frac{\sqrt{h_L}u_L + \sqrt{h_R}u_R}{\sqrt{h_L} + \sqrt{h_R}}\)
        \item \(\tilde{v} = \frac{\sqrt{h_L}v_L + \sqrt{h_R}v_R}{\sqrt{h_L} + \sqrt{h_R}}\)
        \item \(\tilde{h} = \sqrt{h_L h_R}\)
    \end{itemize}
    \item Eigenvalues: \(\tilde{\lambda}_1 = \tilde{u} - \sqrt{gh}\), \(\tilde{\lambda}_2 = \tilde{u}\), \(\tilde{\lambda}_3 = \tilde{u} + \sqrt{gh}\)
    \item Eigenvectors: \(\tilde{\mathbf{K}}_1 = \begin{bmatrix} 1 \\ \tilde{u} - \sqrt{gh} \\ \tilde{v} \end{bmatrix}\), \(\tilde{\mathbf{K}}_2 = \begin{bmatrix} 0 \\ 0 \\ 1 \end{bmatrix}\), \(\tilde{\mathbf{K}}_3 = \begin{bmatrix} 1 \\ \tilde{u} + \sqrt{gh} \\ \tilde{v} \end{bmatrix}\)
    \item Wave strengths: Solve \(\mathbf{U}_R - \mathbf{U}_L = \sum_{k=1}^3 \tilde{\alpha}_k \tilde{\mathbf{K}}_k\)
\end{itemize}

\subsection*{MUSCL Reconstruction}
For second-order accuracy, use MUSCL (Monotonic Upstream-centered Scheme for Conservation Laws):
\begin{itemize}
    \item Reconstruct \(\mathbf{U}\) at cell interfaces using linear interpolation.
    \item Slope limiter (e.g., minmod, superbee) to ensure monotonicity:
    \[
    \mathbf{U}_{i+1/2,L} = \mathbf{U}_i + \frac{1}{2} \phi(\mathbf{r}) (\mathbf{U}_i - \mathbf{U}_{i-1})
    \]
    \[
    \mathbf{U}_{i+1/2,R} = \mathbf{U}_{i+1} - \frac{1}{2} \phi(\mathbf{r}) (\mathbf{U}_{i+2} - \mathbf{U}_{i+1})
    \]
    Where \(\phi(\mathbf{r})\) is the limiter function, and \(\mathbf{r}\) is the ratio of consecutive gradients.
\end{itemize}

\subsection*{Mesh Handling}
\begin{itemize}
    \item Import unstructured mesh (quadrilateral/triangular) from a standard format (e.g., Gmsh .msh).
    \item Store cell connectivity, face normals, and areas.
    \item Compute fluxes across each face using Roe solver and MUSCL-reconstructed states.
\end{itemize}

\end{document}